

% Qué tipo de documento estamos por comenzar:
\documentclass[12pt]{article} 
% Esto es para que el LaTeX sepa que el texto está en español:
\usepackage[spanish]{babel}
\selectlanguage{spanish}
% Esto es para poder escribir acentos directamente:
\usepackage[utf8]{inputenc}
\usepackage[T1]{fontenc}
\usepackage{mathptmx}


%% Asigna un tamaño a la hoja y los márgenes
\usepackage[letterpaper,top=2.4cm,bottom=2.4cm,left=2.4cm,right=2.4cm,marginparwidth=1.5cm]{geometry}

%% Paquetes de la AMS
\usepackage{amsmath, amsthm, amsfonts}
%% Para añadir archivos con extensión pdf, jpg, png or tif
\usepackage{graphicx}
\usepackage[colorinlistoftodos]{todonotes}
\usepackage[colorlinks=true, allcolors=blue]{hyperref}

%% Primero escribimos el título
\title{IMPULSADORES DEL CAMBIO DIGITAL}
\author{J. P. Araiz López\\
  \small 7690-17-9477 Universidad Mariano Gálvez\\
  \small Seminario de Tecnologías de Información\\
  \small jaraizl@miumg.edu.gt
}



\begin{document}
%% Hay que decirle que incluya el título en el documento
\maketitle

%% Aquí podemos añadir un resumen del trabajo (o del artículo en su caso) 
\begin{abstract}
La transformación digital es el proceso mediante el cual las organizaciones integran tecnologías digitales en todos sus aspectos, cambiando fundamentalmente cómo operan y cómo entregan valor a sus clientes. Este proceso es impulsado por varios factores clave que aceleran su adopción y éxito. Entre los principales impulsadores de la transformación digital se encuentran los avances tecnológicos, como la inteligencia artificial y la nube, que ofrecen nuevas oportunidades para innovar y mejorar la eficiencia. La demanda del mercado también juega un papel crucial, ya que los consumidores y las empresas buscan soluciones más rápidas y personalizadas. La competencia en el mercado obliga a las organizaciones a adoptar tecnologías digitales para mantenerse relevantes y competitivas.

\end{abstract}
\maketitle{}

%% Iniciamos "secciones" que servirán como subtítulos
%% Nota que hay otra manera de añadir acentos

\section*{Palabras Clave}
Cambio, Digital, Tecnologías 

\section*{Impulsores de transformación digital.}

\subsection*{Disruptores digitales}
Las historias de éxito de Uber y Lyft que alteran la industria del transporte con su marco de intercambio de viajes bajo demanda se utilizan comúnmente como ilustración de la interrupción digital. Esto se debe a que son un excelente ejemplo de cómo la estrategia y la innovación, habilitadas por la tecnología, pueden influir inesperadamente en el curso de toda la industria. Ahora más que nunca, con la evolución constante y rápida de las tecnologías, los líderes empresariales deben transformar su forma de pensar sobre la competencia y los disruptores de su industria.

\subsection*{Cambio demográfico}
A medida que la jubilación de los baby boomers continúa ganando impulso, las empresas comenzarán a sentir el cambio tanto desde la perspectiva del cliente como desde la perspectiva de la fuerza laboral. A medida que cualquier generación para el caso, hagan la transición a nuevas etapas de la vida y al mundo cambiante de la tecnología para que de esa manera puedan evolucionar. Y si las empresas no son ágiles y no comprenden las necesidades cambiantes de su público objetivo, tendrán dificultades para atraer y retener clientes.

\subsection*{Expectativas del cliente}
La tecnología avanzada ha permitido a los consumidores compartir sus opiniones, experiencias e ideas con las masas con solo tocar un botón. El cliente moderno valora la opinión de extraños sobre la publicidad y tiene altas expectativas para la experiencia general del cliente. Los clientes esperan respuestas rápidas de las marcas, fácil acceso a soluciones y herramientas, opiniones de autoservicio, mensajes personalizados y quieren sentirse siempre valorados en cada punto de contacto de su viaje.



\section*{Barreras de la transformación digital.}

\subsection*{Personal}
Desde puestos de nivel de entrada hasta el C-Suite, toda la fuerza laboral desempeña un papel fundamental en la transformación digital de una empresa.  Según un estudio realizado por Dell e Hitachi, el 56\% de los ejecutivos encuestados cree que el aspecto humano es la clave para una transición exitosa. Ese mismo estudio destaca que un tercio de las compañías declaran que su mayor barrera para la transformación digital es el aspecto humano. Estos hallazgos les dicen a las empresas que la fuerza laboral es crítica para el éxito, pero es un gran desafío para dominar.

\subsection*{Datos y análisis}
Los desafíos que enfrentan las organizaciones cuando se trata de big data es la falta de compresión de la administración y la falta de alineamiento organizacional. Según NewVantage Parterns, más del 85\% de los encuestados informaron que sus empresas han comenzado programas para crear culturas basadas en datos, pero solo el 37\% informa haber tenido éxito hasta el momento. Esto indica que hay conciencia de la importancia de aprovechar los datos, pero solo un tercio de las empresas están aprovechando la big data.

\subsection*{Agilidad operacional}
En los negocios de hoy, la agilidad operativa es conocida por muchos nombres como “lean” o “diseño del pensamiento”. Independientemente como lo llame su empresa, la agilidad operativa es clave para mantenerse al día en el espacio digital. Además de tener una visión clara, una estrategia y un liderazgo empoderado, una organización necesita estar posicionada para adaptarse rápida y eficientemente a los cambios en la tecnología, las tendencias del mercado y las necesidades del cliente.

\section*{Tipos de Software: Impulsores de la Transformación Digital}

\subsection*{Software de Sistema: La Base de la Operatividad}
El software de sistema, también conocido como software de base, es fundamental para el funcionamiento de un computador, incluyendo: 
\begin{itemize}
\item  Sistemas operativos
\item Controladores
\item  Herramientas de diagnóstico
\end{itemize}


\subsection*{Software de Programación: Creando Soluciones Digitales}
El software de programación es un aspecto crucial en el desarrollo de aplicaciones y soluciones digitales. El mismo incluye:
\begin{itemize}
\item  Compiladores
\item Entornos de Desarrollo Integrados (IDEs)
\item  Depuradores
\item Odoo
\end{itemize}

\subsection*{Software de Aplicación: Interacción y Productividad}
Este software se refiere a programas diseñados para realizar tareas específicas en un computador. Es crucial para la interacción del usuario y la productividad, contando con ejemplos que incluyen:
\begin{itemize}
\item  Procesadores de texto
\item Navegadores Web
\item  Hojas de cálculo
\end{itemize}

\subsection*{Software Empresarial y de Productividad}
El software de aplicación específico, como CRM, herramientas de marketing y software educativo, mejoran la productividad y la gestión empresarial al:
\begin{itemize}
\item  Optimizar campañas de Marketing
\item Facilitar la enseñanza
\item  ERP
\end{itemize}


\subsection*{Software Especializado: Adaptándose a Necesidades Específicas}
El software especializado en diferentes industrias, como la médica y de diseño asistido, se adapta a necesidades específicas al:
\begin{itemize}
\item  Optimizar procesos
\item Facilitar el diseño
\item  Automatizar tareas
\end{itemize}



\section*{Observaciones y Comentarios}
Los impulsadores de la transformación digital son factores clave que permiten a las organizaciones adaptarse y prosperar en un entorno cada vez más tecnológico y competitivo. Avances tecnológicos como la inteligencia artificial y la nube proporcionan las herramientas necesarias para innovar y mejorar la eficiencia operativa. La demanda del mercado, impulsada por las expectativas de los clientes por experiencias más personalizadas y rápidas, ejerce una presión constante para que las empresas adopten soluciones digitales. La competencia en el mercado y las políticas públicas que fomentan la digitalización crean un marco favorable para el cambio.

\section*{Conclusiones}
En conclusión, comprender y gestionar los impulsadores de la transformación digital es esencial para cualquier organización que busque prosperar en la era digital. La clave está en integrar estos factores de manera estratégica y adaptativa para maximizar los beneficios y superar los desafíos asociados con la digitalización.

\begin{thebibliography}{9}
\bibitem{TSITD}\textit{Tipos de Software: Impulsores de la Transformación Digital} [En linea]. Disponible en:\\ \url{https://lisit.cl/novedades/tipos-de-software-impulsores-de-la-transformacion-digital/}
\bibitem{ITD8S}\textit{Impulsores de transformación digital.}[En linea]. Disponible en:\\ \url{https://cubymarketer.com/impulsores-y-barreras-a-la-transformacion-digital/#page-content }
\bibitem{QTD}\textit{¿Qué es la transformación digital?}[En linea]. Disponible en:\\ \url{https://www.ibm.com/mx-es/topics/digital-transformation}
\end{thebibliography}
Repositorio git disponible en:\\ \url{https://github.com/JaraizL/SEMINARIO-DE-TECNOLOG-AS.git}
\end{document}