

% Qué tipo de documento estamos por comenzar:
\documentclass[12pt]{article} 
% Esto es para que el LaTeX sepa que el texto está en español:
\usepackage[spanish]{babel}
\selectlanguage{spanish}
% Esto es para poder escribir acentos directamente:
\usepackage[utf8]{inputenc}
\usepackage[T1]{fontenc}
\usepackage{mathptmx}


%% Asigna un tamaño a la hoja y los márgenes
\usepackage[letterpaper,top=2.4cm,bottom=2.4cm,left=2.4cm,right=2.4cm,marginparwidth=1.5cm]{geometry}

%% Paquetes de la AMS
\usepackage{amsmath, amsthm, amsfonts}
%% Para añadir archivos con extensión pdf, jpg, png or tif
\usepackage{graphicx}
\usepackage[colorinlistoftodos]{todonotes}
\usepackage[colorlinks=true, allcolors=blue]{hyperref}

%% Primero escribimos el título
\title{NORMAS APA Y METODOLOGÍA DE LA INVESTIGACIÓN}
\author{J. P. Araiz López\\
  \small 7690-17-9477 Universidad Mariano Gálvez\\
  \small Seminario de Tecnologías de Información\\
  \small jaraizl@miumg.edu.gt
}



\begin{document}
%% Hay que decirle que incluya el título en el documento
\maketitle

%% Aquí podemos añadir un resumen del trabajo (o del artículo en su caso) 
\begin{abstract}
Las normas APA son una guía nos proporcionan ciertas reglas para la elaboración y presentación de trabajos académicos o de investigación, proporcionando una estructura clara que ayuda y facilita la comprensión del trabajo dándole más valor al mismo.
Las metodologías de investigación es un componente esencial para el desarrollo de una investigación sin importar el ámbito en que este se realiza, entender y aplicar correctamente las técnicas permite al investigador realizar un mejor trabajo a su vez proporciona una mayor validez a un trabajo de investigación. 

\end{abstract}
\maketitle{}

%% Iniciamos "secciones" que servirán como subtítulos
%% Nota que hay otra manera de añadir acentos

\section*{Palabras Clave}
Normas, Estructura, Investigación

\section*{Normas APA}

Las normas APA son un conjunto de normas de redacción y presentación que dictan el estilo y estructura de un documento, en forma más básica se puede decir que es un estilo de organizar y presentar un informe, investigación o cualquier documento de una forma más profesional, estas normas no son aplicadas de igual forma ya que existen formas adaptadas de las reglas según la universidad o entidad para la cual se esté realizando el documento a presentar. 

Para la 7ma edición de la norma, se cuentan con algunas novedades las cuales comprenden lo siguiente:
\begin{itemize}
\item Inclusión de DOI: Se recomienda incluir el Digital Object Identifier (DOI) en las referencias siempre que esté disponible.
\item Formato de la Lista de Referencias: Simplificado para hacer más claro el formato de las diferentes fuentes (libros, artículos, sitios web, etc.).
\item Uso de "et al.": Ahora se utiliza para todas las citas con tres o más autores en la primera instancia y posteriormente en todas las citas.
\end{itemize}


\subsection*{Estructura del Documento}
\begin{itemize}
\item Página de título: Incluye el título del trabajo, autor(es), institución y detalles del curso.
\item Resumen:  Un breve resumen del contenido del trabajo, con una extensión máxima de 250 palabras.
\item Cuerpo del documento: Se organiza en sección según la indole del trabajo (introducción, método, resultados, discusión, conclusiones, etc.).
\item Referencias: Lista de las fuentes citadas.
\end{itemize}

\subsection*{Citas y Referencias}
\begin{itemize}
\item Citas en el texto: Se utilizan con paréntesis para citar autores y fechas (Apellido del autor, año).
\item Referencias: Detallan toda la información de cada fuente citada, siguiendo un formato específico para libros, artículos, sitios web, entre otros.
\end{itemize}

\subsection*{Estilo y Formato}
\begin{itemize}
\item  Fuente o tipografía: se utiliza Times New Roman, tamaño 12 .
\item Espaciado: Doble espacio en todo el documento.
\item Márgenes: 2.54 cm (1 pulgada) en todos los lados.
\item Encabezados: Utilizar diferentes niveles de encabezados para organizar el contenido, se puede utilizar hasta el nivel 5 pero se recomienda utilizar hasta el nivel 3.
\end{itemize}

\subsection*{Beneficios de las Normas APA }

Las normas APA aportan un grado de presentación mayor y mas profesional a los documentos acedmicos y de investigación, ya que elevan su calidad al exigir un formato que asegura la claridad y coherencia en la presentación del mismo.


\section*{Metodologia de la Investigación}

La metodología de la investigación es un conjunto de técnicas y procedimientos aplicados de una forma ordenada para la realización de un estudio o investigación, la metodología de investigación busca otorgar validez y sentido a cualquier trabajo de investigación y es aplicable a cualquier área de estudio desde temas sociales hasta científicos.
Podemos decir que una buena metodología de investigación debe tener ciertas características que son:
\begin{itemize}
\item Rigor científico: se deben seguir principios de ciencia, utilizando la observación, creación de hipótesis, recolección y análisis de datos.
\item Claridad y precisión: debe ser clara y precisa en cada etapa de la investigación.
\item Flexibilidad: debe poder adaptarse a cualquier novedad o necesidad del estudio realizado, sin perder su estructura.
\item Relevancia: debe proporcionar relevancia al campo de estudio en que se aplica, aportando al conocimiento existente.
\end{itemize}

La metodología de la investigación se divide en dos categorías generales que son la cualitativa y la cuantitativa, cada una de ellas con sus técnicas y objetivos específicos.

\subsection*{ Investigación Cuantitativa}
Se enfoca en la recolección y análisis de datos numéricos, utilizando herramientas estadísticas para validar hipostasis. Utiliza métodos más comunes como las encuestas, experimentos y análisis estadístico. Este enfoque permite generalizar resultados y a la vez ofreces una precisión y objetividad del análisis.

\subsection*{Investigación Cualitativa}
Se enfoca en la comprensión de fenómenos más complejos a través de datos no numéricos, buscando explorar significados, experiencias y perspectivas. Utiliza métodos como entrevistas, grupos de discusión y observación de objetivos. Este enfoque permite una comprensión más rica y contextualizada ayudando a explorar nuevas teorías de investigación.


\section*{Observaciones y Comentarios}
Las normas APA no son la única estructura para presentación de trabajos académicos y de investigación, pero estas proporcionan una mejor presentación y claridad a los trabajos, ya que sus reglas se pueden personalizar según la necesidad de la entidad en la cual se está trabajando.
La metodología de investigación es más una guía a la forma de realizar una investigación, estas no son reglas fijas y además se puede utilizar diferentes metodologías siempre y cuando estas ayuden a realizar un trabajo de mejor validez.


\section*{Conclusiones}
Las normas APA y la metodología de la investigación son dos herramientas que se complementan ya que ambas ayudan a la realización de trabajos más profesionales y de mejor calidad, a la vez que dan una mayor validez al trabajo de investigación realizado.

\begin{thebibliography}{9}

\bibitem{guiaapa}\textit{Guía Normas APA 7ma Edición} [En linea]. Disponible en:\\ \url{https://normas-apa.org/ }
\bibitem{formaapa}\textit{Formato de las Normas APA 7ma edición} [En linea]. Disponible en:\\ \url{https://normasapa.in/}
\bibitem{FC}\textit{Metodología de la Investigación} [En linea]. Disponible en:\\ \url{https://www.significados.com/metodologia-de-la-investigacion/ }
\bibitem{UdlA}\textit{Metodología de la Investigación} [En linea]. Disponible en:\\ \url{https://programas.uniandes.edu.co/blog/metodologia-de-la-investigacion }
\end{thebibliography}

Repositorio git disponible en:\\ \url{https://github.com/JaraizL/SEMINARIO-DE-TECNOLOG-AS.git}
\end{document}