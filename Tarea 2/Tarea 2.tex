

% Qué tipo de documento estamos por comenzar:
\documentclass[12pt]{article} 
% Esto es para que el LaTeX sepa que el texto está en español:
\usepackage[spanish]{babel}
\selectlanguage{spanish}
% Esto es para poder escribir acentos directamente:
\usepackage[utf8]{inputenc}
\usepackage[T1]{fontenc}
\usepackage{mathptmx}


%% Asigna un tamaño a la hoja y los márgenes
\usepackage[letterpaper,top=2.4cm,bottom=2.4cm,left=2.4cm,right=2.4cm,marginparwidth=1.5cm]{geometry}

%% Paquetes de la AMS
\usepackage{amsmath, amsthm, amsfonts}
%% Para añadir archivos con extensión pdf, jpg, png or tif
\usepackage{graphicx}
\usepackage[colorinlistoftodos]{todonotes}
\usepackage[colorlinks=true, allcolors=blue]{hyperref}

%% Primero escribimos el título
\title{BUENAS PRÁCTICAS PARA EL DESARROLLO DE APLICACIONES ÁGILES}
\author{J. P. Araiz López\\
  \small 7690-17-9477 Universidad Mariano Gálvez\\
  \small Seminario de Tecnologías de Información\\
  \small jaraizl@miumg.edu.gt
}



\begin{document}
%% Hay que decirle que incluya el título en el documento
\maketitle

%% Aquí podemos añadir un resumen del trabajo (o del artículo en su caso) 
\begin{abstract}
El desarrollo ágil de aplicaciones se basa en el manifiesto ágil que resume los valores y principios para aplicar correctamente la metodología, en este artículo conoceremos un poco sobre estos valores y principios, también algunas de las mejores prácticas para el desarrollo ágil y por qué se deben de realizar.

\end{abstract}
\maketitle{}

%% Iniciamos "secciones" que servirán como subtítulos
%% Nota que hay otra manera de añadir acentos

\section*{Palabras Clave}
Mejora, Comunicación, Continuo 

\section*{Metodología Ágil}

Para comenzar debemos conocer a que nos referimos cuando se habla de desarrollo ágil, esto se refiere a un concepto de gestión de proyectos que busca hacer dividir el proyecto en diferentes fases las cuales se irán entregando de forma continua, el desarrollo o metodología ágil hace mucho énfasis en la colaboración y la mejora continua.

Según el manifiesto ágil, existen 4 valores que promueven los modelos de organización enfocados principalmente en las personas y la colaboración entre ellas, estos valores son: 
\begin{itemize}
\item Individuos e interacciones sobre procesos y herramientas.
\item Software funcionando sobre documentación extensiva.
\item Colaboración con el cliente sobre negociación contractual.
\item Respuesta ante el cambio sobre seguir un plan.
\end{itemize}

Esta metodología también  nos presenta 12 principios que ayudara a la correcta aplicación de la metodología ágil, estos principios buscan satisfacer al cliente mediante la entrega constante y eficaz, los principios son:
\begin{itemize}
\item Satisfacer al cliente de forma constante.
\item Los cambios son bienvenidos.
\item El trabajo se realiza en fases con tiempos cortos definidos.
\item Medir el progreso.
\item Continuidad del proyecto.
\item Trabajo organizado.
\item Comunicación entre todos los interesados.
\item Motivar y generar confianza entre los miembros del proyecto.
\item Excelencia técnica y un buen diseño.
\item Las tareas deber ser los más sencillas posible.
\item Equipos autogestionados.
\item Adaptación a las circunstancias.
\end{itemize}

\section*{Buenas Practicas para el Desarrollo Ágil }
El desarrollo ágil se ha convertido en un estándar para la realización de proyectos ya que se maneja de una forma flexible y eficiente, por lo que existen una serie de buenas prácticas que ayudan a mejorar y garantizar un desarrollo de aplicaciones ágil.
Definir claramente los requisitos, esto se podría decir que es una parte fundamental del desarrollo ágil, esto debido a que se debe tener una visión inicial clara de lo que se debe realizar.\\ 

Entregas frecuentes, se deben definir iteraciones cortas para el desarrollo ágil, ya que esto ayuda a recibir una retroalimentación continúa ayudando a que el proyecto evolucione según las necesidades del cliente.\\ 

Comunicación constante, la comunicación efectiva entre cada miembro del equipo y los interesados es importante ya que esto ayuda a mantener informados a todos sobre el avance o dificultades que se tengan en el proyecto.\\ 

Pruebas automatizadas, es recomendable realizar continuamente pruebas automatizadas para garantizar que el proyecto sea funcional y minimice la cantidad de errores en el mismo.\\ 

Refinamiento constante del producto, es importante mantener relevante y priorizado el feedback que se recibe para asegurar que el equipo este siempre trabajando en lo que más relevante del proyecto.\\ 

Documentación eficiente, es importante realizar la documentación de la forma más ligera, pero a la vez que sea lo más eficaz posible, teniendo como objetivo, proporcionar la suficiente información para que el equipo pueda trabajar.\\ 

Involucrar al cliente, la participación constante del cliente es fundamental para que el proyecto final cumpla con todas sus necesidades.\\ 

Trabajo en equipo, es importante que cada miembro tenga la autonomía para tomar decisiones y pueda organizar su trabajo de forma eficiente, pero también es importante promover la comunicación y colaboración entre todos los miembros del equipo\\ 

Mejora continua, es importa reflexionar después de cada entrega sobre lo que se hizo bien y lo que se debe mejorar, también es importante mantener una capacitación constante del equipo.


\section*{Observaciones y Comentarios}
El desarrollo ágil se basa en un manifiesto el cual nos da una orientación del objetivo de la metodología ágil, pero no es una ley o norma, ya que esto se puede adaptar según la necesidad del proyecto. El desarrollo ágil suele atribuirse a la metodología SCRUM, pero este no es el único enfoque puesto que existen otros enfoques que también promueven un enfoque ágil, como lo es DevOps o Kanban.


\section*{Conclusiones}
El desarrollo ágil se ha convertido en un pilar importante en la realización de proyectos de desarrollo de software, por lo que se hace necesario entender y conocer algunas de las practicas que se realizan con esta metodología.

\begin{thebibliography}{9}
\bibitem{UdlA}\textit{¿El manifiesto ágil sigue estando disponible?} [En linea]. Disponible en:\\ \url{https://www.atlassian.com/es/agile/manifesto}
\bibitem{FC}\textit{Principios detrás del Manifiesto Ágil} [En linea]. Disponible en:\\ \url{https://agilemanifesto.org/principles.html }
\bibitem{pega}\textit{Buenas prácticas para el desarrollo de aplicaciones ágiles} [En linea]. Disponible en:\\ \url{https://academy.pega.com/es/topic/agile-development-best-practices/v5}
\bibitem{cognodata}\textit{12 principios de la metodología agile en el desarrollo de proyectos} [En linea]. Disponible en:\\ \url{https://www.cognodata.com/blog/principios-metodologia-agile-desarrollo-proyectos/}


\end{thebibliography}

Repositorio git disponible en:\\ \url{https://github.com/JaraizL/SEMINARIO-DE-TECNOLOG-AS.git}
\end{document}